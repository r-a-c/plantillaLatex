\documentclass[12pt,english,spanish,noprefix,refpage]{book}
\usepackage{xcolor}
%\usepackage[onehalfspacing]{setspace}
\usepackage{pdfpages}
\usepackage{amsmath}
\usepackage{nicematrix}
\usepackage{setspace}
\usepackage{babel}
\usepackage{graphics}
\usepackage{fancyhdr}
\usepackage{lscape}
\usepackage{adjustbox,lipsum}
\fancyhf{}
\pagestyle{fancy}
\newcommand{\mychangefont}{%
	\fontsize{8}{10}\selectfont
}
\fancyhead[LE,RO]{\mychangefont \thepage}
\fancyhead[RE,LO]{\mychangefont \leftmark}


\usepackage{babel}
\addto\shorthandsspanish{\spanishdeactivate{~<>}}
\addto\captionsenglish{\renewcommand{\chaptername}{Capítulo}}

\usepackage[utf8]{inputenc}
\usepackage[hidelinks]{hyperref}
\usepackage{float}


\usepackage[tablename=Tabla,figurename=Imagen]{caption}


\usepackage{listings}
\addto\captionsenglish{\renewcommand{\lstlistingname}{Código}}
\addto\captionsspanish{\renewcommand{\lstlistingname}{Código}}

\usepackage{listingsutf8}
\lstset{inputencoding=utf8/latin1}
\lstset{inputpath=files/}
\lstset{breaklines=true,basicstyle=\footnotesize}
\usepackage{graphicx} %Necesario para el siguiente comando
\graphicspath{ {./images/} }
\usepackage{multirow}
\usepackage{siunitx}
\sisetup{
	round-mode          = places, % Redondea números
	round-precision     = 2, % a 2 lugares
}


\usepackage[sorting=none,backend=biber]{biblatex}
\addbibresource{biblio.bib}
\nocite{*}
\renewcommand{\bibname}{Bibliografía}

\newcommand{\leadingzero}[1]{\ifnum #1<10 0\the#1\else\the#1\fi}
\newcommand{\mytoday}{\the\year-\leadingzero{\month}-\leadingzero{\day}}
\DeclareFieldFormat{urldate}{Último acceso en #1}

\DeclareSourcemap{
	\maps[datatype=bibtex]{
		\map[overwrite=true]{
			\pertype{online}
			\step[fieldset=urldate,fieldvalue={\mytoday} ]
		}
	}
}
\usepackage[T1]{fontenc}
\usepackage{hyphenat}
\hyphenation{Ca-ta-lo-gue-DB gra-nu-la-ri-dad exis-ta mí-ni-mo prio-ri-dad má-xi-mo ca-rac-te-ri-za-ción mo-de-los de-sa-rro-lla-do-res des-cri-to orien-ta-ción pro-pues-tos ca-rac-te-rís-ti-cas des-a-co-pla-mien-to u-san-do mo-der-nas ge-ne-ral-men-te res-pues-ta do-cu-men-ta-ción pla-ni-fi-ca-dor co-rrec-ta-men-te des-plie-gue vir-tua-li-za-ción sto-ra-ge va-rios con-si-de-ra-ción he-rra-mien-ta con-ti-nua-ción mo-de-lo ME-TRI-CA MÉ-TRI-CA cum-pli-mien-to ta-reas per-so-na-les}

\usepackage{verbatim}

\usepackage{nomencl}
\makenomenclature


\setcounter{tocdepth}{4}
\setcounter{secnumdepth}{3}
\begin{document}

	%	\nomenclature{$POR$}{ }


	\begin{titlepage}\thispagestyle{empty}%
	\begin{minipage}[c][0.99\textheight][t]{0.95\columnwidth}%
		%\vspace{1cm}
		\begin{center}
			%\includegraphics[width=2cm]{images/logo}
			\par\end{center}
		\bigskip{}
		\begin{singlespace}
			\begin{center}
				{\large{}UNIVERSIDAD  INSTITUCIÓN}\vspace{0.7cm}
				\par\end{center}
			\begin{center}
				{\large{}ESCUELA TÉCNICA SUPERIOR DE INGENIERÍA INFORMÁTICA}\vspace{0.5cm}
				\par\end{center}
			\begin{center}
				TITULO {\Large{}\vspace{0.8cm}
				}
				\par\end{center}{\Large \par}
		\end{singlespace}
		\begin{center}
			\textbf{\Large{}MÁS TITULO \\}

			\par\end{center}{\Large \par}
		\vfill{}
		\begin{flushleft}
			\qquad{}Raúl Álvarez de Celis\bigskip{}
			\par\end{flushleft}
		\begin{flushleft}

			\par\end{flushleft}
		\begin{flushleft}
			\qquad{}Curso: 2021-2022, convocatoria Febrero
			\par\end{flushleft}
		\bigskip{}
		%
	\end{minipage}

\end{titlepage}
\thispagestyle{empty}
	%	\include{primerapagina}\thispagestyle{empty}
	%	\include{preambulo}\thispagestyle{empty}

	\cleardoublepage{}
	\renewcommand{\contentsname}{Índice}
	\tableofcontents{}
	\cleardoublepage{}
	%\renewcommand{\listfigurename}{Índice de imágenes}
	%\listoffigures
	%	\renewcommand{\listtablename}{Índice de tablas}
	%	\listoftables
	%\renewcommand{\lstlistlistingname}{Índice de código y algoritmos}
	%\lstlistoflistings
	%	\renewcommand{\nomname}{Abreviaturas}
	%	\printnomenclature
	%\cleardoublepage{}



	\chapter{Propuesta título 1}


\section{Introducción}\thispagestyle{empty}
%	\include{pregunta2}\thispagestyle{empty}
%	\include{pregunta3}\thispagestyle{empty}
%	\include{pregunta4}\thispagestyle{empty}

	\renewcommand{\appendixname}{Apéndice}
	\appendix
	\cleardoublepage{}
	\chapter{Apendice 1}


\section{Introducción}





	\printbibliography[heading=bibintoc,title={Bibliografía}]
	\newpage\null\thispagestyle{empty}\newpage

	%\include{huevo}\thispagestyle{empty}
\end{document}